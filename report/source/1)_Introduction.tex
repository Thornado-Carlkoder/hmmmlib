\section{Introduction}
Hidden Markov Models\cite{baum1966} are a group of powerful probabilistic models for modeling sequential data with a hidden underlying structure. They are widely used in the field of bioinformatics for gene finding and gene prediction. The reason for their popularity lies in their accuracy, robustness and simplicity.
When analyzing larger quantities of data with an increasingly complex underlying structure, the efficiency of Hidden Markov Models is a limiting factor. This limiting factor is rooted in the algorithms used for evaluating, decoding and training the Hidden Markov Models. All algorithms except the Viterbi algorithm rely strongly on the efficiency of two main algorithms: the Forward and the Backward algorithms.
In this report we propose a solution to the efficiency problem of Hidden Markov Models by optimizing the Forward and Backward algorithms using linear algebra. Our solution is implemented in two libraries, one for the $C$\todo{andre steder er det formateret med textt. } programming language and one for the $Python$ programming language utilizing the implementations in the $C$ library.
%Nej det er en paper convention hvor beskrivelsen af resten af rapporten er i lille da den enligt ikke er en del af introduktionen.

{\small The rest of this report is structured as follows: We will give an introduction to Hidden Markov Models and their associated algorithms in \textbf{section \ref{sec:HMM}}. Then we will present our optimizations of the Forward and the Backward algorithms in \textbf{section \ref{sec:ForwardBackward}}. Then we will present our library and its underlying architecture in \textbf{section \ref{sec:hmmmlib}}. Then we will present our results from conducting experiments with our library and see if we achieved a speed up compared to a conventional implementation in \textbf{section \ref{sec:expriments}}. We will finish the report with a conclusion in \textbf{section \ref{sec:conclusion}}.}
