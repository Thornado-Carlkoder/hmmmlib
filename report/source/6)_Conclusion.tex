\section{Conclusion}\label{sec:conclusion}


In this project report we have made a linear algebra formulation of the Forward and Backward algorithms with the hypothesis that this would increase the efficiency of these to algorithms, as well as the Baum-Welch and the Posterior decoding algorithms when implemented. We have implemented these algorithms in a conventional manner and the linear algebra formulation using BLAS. With the latter we have conducted experiments that supports our hypothesis. We have also made an implementation of the Forward and the Backward algorithms which takes advantage of the sparse instances of an HMM using RSB. We have made experiments with results supporting our hypothesis that the sparse instances of an HMM allows for an even greater speed up.
From this project report and our experiments we can conclude that the linear algebra implementation of the Forward and Backward algorithms yields a significant increase in speed not only for the latter two but for all the algorithms, that uses them as an subroutine. We can also conclude that an even greater increase in speed can be achieved when the instance of the HMM is sparse.


% In this project report we have made a linear algebra formulation of the Forward and the Backward algorithms with the hypothesis that this would increase the efficiency of these to algorithms as well as the Baum-Welch and Posterior decoding algorithms when implemented. We have implemented these algorithms in a conventional manner and the linear algebra formulation using BLAS. With the latter we have conducted experiments which results support our hypothesis. We have as well made an implementation of the Forward and the Backward algorithms which takes advantage of the sparse instances of an HMM using RSB. We have made experiments with results that support our hypothesis that the sparse instances of an HMM allows for a even greater speed for the sparse optimized implementations.
% From this project report and our experiments we can conclude that the linear algebra implementation of the Forward and Backward algorithms yields a significant increase in speed not only for the latter two but for all the algorithms that use these algorithms as a subroutine. We can also conclude that an additional increase in speed can be achieved when the instance of the HMM is sparse.


